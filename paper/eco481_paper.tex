\documentclass{eco481_paper}
\addbibresource{bibliography.bib}
\begin{document}
\begin{center}
    \sffamily\Large\textbf{Commuting and Health}
\end{center}
\section*{Abstract}
The correlation between commute time and worse health outcomes is well-known, but what about the causal effect of commute duration on one's health? Using pooled cross-sectional data from the American Time Use Survey for 2009-2015, I present evidence to confirm the correlation between commute duration and health outcomes and behaviours. I attempt an instrumental variables analysis using Uber's staggered entry across U.S. Metropolitan Statistical Areas between 2011 and 2015 as an instrument for commute duration. My results show no significant causal effect between commute duration and health outcomes.

\section*{Introduction}
How does commuting affect our health? Spending a longer time commuting may induce stress, anxiety, and tension, and it may reduce the time available for other activities important to health. Commuting is an especially important issue due to its scope and ubiquity: most workers commute, and many workers spend a significant portion of their day doing so. In this paper, I use pooled cross-sections of the American Time Use Survey (ATUS) to answer the question: How does time spent commuting affect a person’s health? I find that there is no significant relationship between time spent commuting and most self-reported measures of health, except for high blood pressure. There is, however, a large and significant negative correlation between time spent commuting and time spent engaging in healthful behaviours like exercising, sleeping, and preparing food. I also attempt an instrumental variables analysis using Uber’s staggered entry across U.S. urban markets as an instrument for commuting time, which yields insignificant results.

Existing literature suggests that there are numerous health costs to long commutes. Longer car commutes are positively correlated with a variety of immediate negative health outcomes like high blood pressure, tension, and tiredness (Lyons \& Chatterjee, 2008, p. 185). A study in the United Kingdom using British Household Panel Survey data finds that there is no significant relationship between commuting time and most objective health measures (Künn-Nelen, 2016, p. 992). Longer commute times, however, are negatively correlated with subjective health measures (Künn-Nelen, 2016, p. 992). In Sweden, health outcomes were monotonically decreasing with commute time for public transport users. For car users, commute time has a negative effect of up to 60 minutes and a positive effect after (Hansson et al., 2011, p. 1). 

Little is understood about the long-run effects of commuting on health (Lyons \& Chatterjee, 2008, p. 186). In this paper, I will attempt to replicate the findings on the short-run impacts of commuting on health in the United States context. I also begin to uncover the long-run effects of commuting on health by investigating health behaviours.

\section*{Data and Summary Statistics}

I use data from the ATUS generated using the Integrated Public Use Microdata Series (IPUMS) ATUS Extract Builder tool (Flood et al., 2023). The ATUS is a representative annual survey of the amount of time American individuals spend doing daily activities. I choose to use the amount of time spent commuting (“Travel related to work”) as my independent variable. I include time spent commuting via private vehicle only. I use two sets of dependent variables as proxies for health. The first measures self-reported health outcomes as measured by the ATUS Well-Being Module (WBM), which collects information on the health and well-being of each respondent. The second measures health behaviours through time use. 

The variables I chose as measures for self-reported health outcomes were an indicator variable for high blood pressure in the past five years, an indicator variable for the usage of pain medication the day before taking the questionnaire, and a four-level factor variable for feeling well-rested. For the well-rested variable, I recode it to a binary variable so that the top two levels (very and somewhat) are treated as “yes” and the bottom two levels (“a little” and “not at all”) are treated as “no.” I do this for ease of analysis. For measures of health behaviours, I use the amount of time a respondent spends sleeping, preparing food or drink, and participating in exercise. 

For my instrumental variables analysis, I use data on the entry and exit of Uber, a ride-sharing company, in various U.S. Metropolitan Statistical Areas (MSAs) between 2009 and 2015 (Berger et al., 2018). In this period, Uber entered 42 MSAs starting in 2011 out of a total of 206 MSAs for which I have time-use data. Figure \ref*{fig:uber_entry} shows Uber’s entry across MSAs over time. I crosswalk the ATUS and Uber data by using the Bureau of Labor Statistics Quarterly Census of Employment and Wages County-MSA crosswalk (Bureau of Labor Statistics, 2022). For counties with fewer than 100,000 people, the ATUS reports Combined MSA instead of the county, so I impute MSA based on Combined MSA. After merging ATUS and Uber data, 13,181 observations remain from 2009 to 2015. The ATUS WBM data is only collected for 2010, 2012, and 2013, so I have 5291 observations for self-reported health outcomes.

I present summary statistics for commute duration, sleep duration, well-rested status, pain medication status, and high blood pressure status in Table \ref{table:sum_stat}. The mean commute length is approximately 45 minutes per day. Most Americans have fairly short commutes: Figure \ref*{fig:hist_travel_work} shows that the distribution of commuting durations is clustered to the left. On average, in-sample respondents slept for 7.8 hours, prepared food and drink for 18 minutes, and participated in sports and exercise for 11 minutes. Health indicators (well-rested, high blood pressure, and pain medication) are coded as 1 for “no” and 2 for “yes.” Health indicators show that 30\% of those surveyed reported being well-rested the day before the survey, 78\% of those surveyed had high blood pressure in the past five years, and 78\% of those surveyed reported using pain medication the day before the survey. 

\datatable{sum_stat.tex}{sum_stat}{Summary statistics for commute duration, various health behaviours, and various health indicators.}

\section*{Commute duration and health outcomes}
The top three panels of Figure \ref*{fig:binscatter_combined} present three binned scatterplots showing the relationship between commute duration and health outcomes. There is a negative correlation between commute duration and feeling well-rested. There is a slightly negative correlation between high blood pressure and commute duration and very little correlation between pain medication usage and commute duration. Taken together, it seems that those with longer commutes are less healthy than those with shorter commutes. 

This correlation is likely influenced by the bidirectional relationship between commuting duration and health. The causal effect of commuting time on health is likely negative: spending more time commuting may introduce stressors to one’s health. The selection effect of commuting time, however, may induce those with worse pre-existing health conditions to choose workplaces closer to where they live or live closer to their workplace. In Table \ref{table:all_tables} Columns (1) to (3), I present regression results detailing the relationship between commute duration and various health indicators. In each regression, I attempt to control for variables correlated with commute duration and underlying health conditions, such as age, income, education, and race. After controlling for these characteristics, the results for feeling well-rested and pain medication are small and statistically insignificant. The effect on high blood pressure is small and precisely estimated; a one-hour increase in commute time is associated with a 0.04 percentage point decrease in the chance of having high blood pressure.

\newfigure[1]{binscatter_combined.png}{binscatter_combined}{Relationship between various health indicators and commute duration.}

\section*{Commute duration and health behaviours}
Another explanation for the insignificant relationship between commute duration and the chosen health indicators is that commute length may have long-run causal effects on health that cannot be captured in this longitudinal dataset. That is, people who have long commutes now may only experience worse health outcomes years or decades later. To overcome this, I examine the effect of commute duration on sleep duration, time spent preparing food, and time spent participating in exercise.  

The bottom three panels of Figure \ref*{fig:binscatter_combined} show the relationship between my chosen health behaviours and commute duration. There is a clear negative correlation between all three behaviours and commute duration, which indicates that those with longer commutes have more unhealthy behaviours. This is confirmed by Table Columns (4) to (6), which show a statistically significant relationship between commute duration and each health behaviour. These results are unsurprising; given that people have a fixed amount of time per day, more time spent commuting requires that people spend less time doing other activities. 

This estimate, however convincing, cannot be treated as causal. One issue is reverse causality: perhaps those who need less sleep to feel healthy will be more likely to choose longer commutes. Omitted variables bias could also play a role. Unobserved health conditions may make it more difficult for respondents to get enough sleep and reduce their commute times. In the case that these estimates are causal, I can treat these estimates as a substitution effect. An additional minute of commuting is associated with a 0.27-minute decrease in time spent asleep, a 0.04-minute decrease in time spent preparing food, and a 0.06-minute decrease in time spent exercising.

\section*{IV Estimation}
To uncover the causal effects of commuting on health, I use the arrival of Uber into local markets as an exogenous shock to commute times. Previous literature has found that Uber increases traffic congestion (Schaller, 2021; Tarduno, 2021), which should in turn increase commute length. I argue that the entry decision of Uber into MSAs is exogenous to the three outcomes I choose to measure: time spent sleeping, time spent preparing food, and time spent exercising. Uber’s entry decision as a profit-maximizing firm means that it enters areas with high potential ridership, which likely do not differ on observable and unobservable health characteristics. Furthermore, Uber’s aggressive expansion was not targeted towards improving people’s health.

Figure \ref*{fig:commute_uber} suggests that Uber entry is associated with higher commute times. In Table \ref*{table:iv_tables} Column (1), I show that the first-stage effect of Uber on commute times is robust. The entry of Uber is associated with a 2.8-minute increase in commute times, which is statistically significant at the p\=0.05 level. A joint test of significance yields an F-statistic of 22.26. The reduced-form results, however, are large and statistically significant. Notably, the coefficients are positive unlike the OLS coefficients reported in Table \ref*{table:all_tables}. This counterintuitive result may suggest that Uber is not a valid instrument; perhaps Uber creates time savings that allow people to spend more time on health-enhancing behaviours. 

\section*{Conclusion}
In this paper, I used U.S. time use microdata to find some evidence of a negative association between commute time and self-reported health measures. Specifically, longer commutes are associated with higher blood pressure but not with feeling more tired or pain medication usage. This somewhat contradicts the finding that longer commutes are associated with a variety of negative health outcomes in the short run (Lyons \& Chatterjee, 2008, p. 185). I find a large and statistically significant relationship between health behaviours (sleep, food preparation, and exercise), and commute time. This may be evidence of a substitution effect between time spent commuting and time spent in healthful behaviours, but it may also reflect diverse time-use choices across different respondents. Finally, I use an instrumental variable approach to test the causal effect of commuting on health by leveraging the staggered entry of Uber, which affects commute times. The results are inconclusive, which may be because Uber's entry does not satisfy the exclusion restriction. In future work, it may be better to switch to a panel dataset and attempt a differences-in-differences analysis to uncover the causal effect of commuting on health.


(1767 words)

\newpage
\printbibliography

\newpage
\section*{Appendix}
\newfigure[0.8]{hist_travel_work.png}{hist_travel_work}{Distribution of commute times.}
\newfigure[0.8]{commute_uber.png}{commute_uber}{Commute duration grouped by Uber status (coded as 1 when Uber is active in respondent's MSA of residence during the year of response; 0 otherwise.)}
\newfigure[0.8]{uber_entry.png}{uber_entry}{aaaa}
\newgeometry{left=1cm, right=1cm}
\landscape
\datatable{all_tables.tex}{all_tables}{Regression results showing the relationship between commute duration and various health indicators.}
\datatable{iv_tables.tex}{iv_tables}{Instrumental variable regression results showing the relationship between commute duration and various health indicators. Uber entry serves as an instrument for commute time in columns (2) to (4)}
\end{document}